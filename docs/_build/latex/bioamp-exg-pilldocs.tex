%% Generated by Sphinx.
\def\sphinxdocclass{report}
\documentclass[a4paper,10pt,english,oneside]{sphinxmanual}
\ifdefined\pdfpxdimen
   \let\sphinxpxdimen\pdfpxdimen\else\newdimen\sphinxpxdimen
\fi \sphinxpxdimen=.75bp\relax
\ifdefined\pdfimageresolution
    \pdfimageresolution= \numexpr \dimexpr1in\relax/\sphinxpxdimen\relax
\fi
%% let collapsible pdf bookmarks panel have high depth per default
\PassOptionsToPackage{bookmarksdepth=5}{hyperref}

\PassOptionsToPackage{warn}{textcomp}
\usepackage[utf8]{inputenc}
\ifdefined\DeclareUnicodeCharacter
% support both utf8 and utf8x syntaxes
  \ifdefined\DeclareUnicodeCharacterAsOptional
    \def\sphinxDUC#1{\DeclareUnicodeCharacter{"#1}}
  \else
    \let\sphinxDUC\DeclareUnicodeCharacter
  \fi
  \sphinxDUC{00A0}{\nobreakspace}
  \sphinxDUC{2500}{\sphinxunichar{2500}}
  \sphinxDUC{2502}{\sphinxunichar{2502}}
  \sphinxDUC{2514}{\sphinxunichar{2514}}
  \sphinxDUC{251C}{\sphinxunichar{251C}}
  \sphinxDUC{2572}{\textbackslash}
\fi
\usepackage{cmap}
\usepackage[T1]{fontenc}
\usepackage{amsmath,amssymb,amstext}
\usepackage{babel}


\usepackage{charter}


\usepackage[Bjarne]{fncychap}
\usepackage{sphinx}
\sphinxsetup{verbatimwithframe=false,VerbatimColor={HTML}{f0f2f4},InnerLinkColor={HTML}{2980b9},warningBgColor={HTML}{e9a499},warningborder=0pt,HeaderFamily=\rmfamily\bfseries}
\fvset{fontsize=auto}
\usepackage{geometry}


% Include hyperref last.
\usepackage{hyperref}
% Fix anchor placement for figures with captions.
\usepackage{hypcap}% it must be loaded after hyperref.
% Set up styles of URL: it should be placed after hyperref.
\urlstyle{same}

\addto\captionsenglish{\renewcommand{\contentsname}{Contents:}}

\usepackage{sphinxmessages}
\setcounter{tocdepth}{1}

% LaTeX documentation preamble
%
% Copyright (c) 2021 Nordic Semiconductor ASA
% SPDX-License-Identifier: Apache-2.0

\usepackage[some]{background}
\usepackage{sectsty}

\definecolor{bg-color}{HTML}{333f67}

\setcounter{tocdepth}{2}

\addto\captionsenglish{\renewcommand{\contentsname}{Table of contents}}

\allsectionsfont{\color{bg-color}}

\title{BioAmp\sphinxhyphen{}EXG\sphinxhyphen{}Pill Docs}
\date{Jul 07, 2022}
\release{0.1.0\sphinxhyphen{}beta}
\author{UpsideDownLabs}
\newcommand{\sphinxlogo}{\vbox{}}
\renewcommand{\releasename}{Release}
\makeindex
\begin{document}

\ifdefined\shorthandoff
  \ifnum\catcode`\=\string=\active\shorthandoff{=}\fi
  \ifnum\catcode`\"=\active\shorthandoff{"}\fi
\fi

\pagestyle{empty}
% LaTeX documentation title page
%
% Copyright (c) 2021 Nordic Semiconductor ASA
% SPDX-License-Identifier: Apache-2.0

\makeatletter
\newgeometry{top=3cm,left=0cm,right=0cm,bottom=2.5cm}

\backgroundsetup{
    scale=2.3,
    contents={\sphinxlogo},
    opacity=0.2,
    angle=0,
    position={0.25\textwidth,-0.4\textheight}
}

\BgThispage

\begin{minipage}{2cm}
    \color{bg-color} \rule{2cm}{1.7cm}
\end{minipage}
\hspace{0.2cm}
\begin{minipage}{3cm}
    \sphinxlogo
\end{minipage}
\hspace{0.2cm}
\begin{minipage}{15cm}
    \Huge \textbf{\@title}\\
    \LARGE \py@release\releaseinfo
\end{minipage}

\vspace{21cm}

\begin{flushright}
    \begin{minipage}{7cm}
        \large \@author\\\@date
    \end{minipage}
    \begin{minipage}{1.5cm}
        \color{bg-color} \rule{1.5cm}{1.3cm}
    \end{minipage}
\end{flushright}

\restoregeometry
\makeatother
\pagestyle{plain}
\sphinxtableofcontents
\pagestyle{normal}
\phantomsection\label{\detokenize{index::doc}}
\noindent\sphinxincludegraphics{{EXG-Pill-GIF-banner}.gif}

\begin{DUlineblock}{0em}
\item[] 
\end{DUlineblock}

\begin{DUlineblock}{0em}
\item[] 
\end{DUlineblock}



\begin{DUlineblock}{0em}
\item[] 
\end{DUlineblock}

\sphinxstepscope


\chapter{Introduction}
\label{\detokenize{introduction/index:introduction}}\label{\detokenize{introduction/index::doc}}
\sphinxAtStartPar
BioAmp EXG Pill is a small (2.54 X 1.00 cm) and elegant Analog Front End (AFE) board for BioPotential signal acquisition that you can use with any 5v Micro Controller Unit (MCU) with an ADC. It is capable of recording publication grade BioPotential signals like ECG, EMG, EOG, and EEG without the inclusion of any dedicated Hardware/Software filter, see \sphinxhref{https://www.youtube.com/watch?v=-G3z9fvQnuw}{BioAmp EXG Pill v0.7 intro video} for more info. The v1.0 of BioAmp EXG pill provides even more flexibilty with configuration option for Gain, BandPass, Filter, and Electrodes.

\begin{figure}[htbp]
\centering
\capstart

\noindent\sphinxincludegraphics{{BioAmp-EXG-Pill-Assembled}.jpg}
\caption{BioAmp\sphinxhyphen{}EXG\sphinxhyphen{}Pill\sphinxhyphen{}Assembled}\label{\detokenize{introduction/index:id2}}\end{figure}

\begin{DUlineblock}{0em}
\item[] 
\end{DUlineblock}

\begin{figure}[htbp]
\centering
\capstart

\noindent\sphinxincludegraphics[width=550\sphinxpxdimen]{{bioamp-exg-pill-board-connection}.jpg}
\caption{BioAmp\sphinxhyphen{}EXG\sphinxhyphen{}Pill\sphinxhyphen{}Connections}\label{\detokenize{introduction/index:id3}}\end{figure}

\begin{figure}[htbp]
\centering
\capstart

\noindent\sphinxincludegraphics[width=550\sphinxpxdimen]{{bioamp-exg-pill-board-configuration}.jpg}
\caption{BioAmp\sphinxhyphen{}EXG\sphinxhyphen{}Pill\sphinxhyphen{}Configuration}\label{\detokenize{introduction/index:id4}}\end{figure}

\begin{DUlineblock}{0em}
\item[] 
\end{DUlineblock}

\begin{figure}[htbp]
\centering
\capstart

\noindent\sphinxincludegraphics[width=600\sphinxpxdimen]{{Basic-Circuit}.jpg}
\caption{BioAmp\sphinxhyphen{}EXG\sphinxhyphen{}Pill Basic\sphinxhyphen{}circuit}\label{\detokenize{introduction/index:id5}}\end{figure}

\begin{figure}[htbp]
\centering
\capstart

\noindent\sphinxincludegraphics[width=600\sphinxpxdimen]{{bioamp-exg-pill-electromyograph}.jpg}
\caption{BioAmp\sphinxhyphen{}EXG\sphinxhyphen{}Pill Electomyograph}\label{\detokenize{introduction/index:id6}}\end{figure}

\begin{figure}[htbp]
\centering
\capstart

\noindent\sphinxincludegraphics[width=600\sphinxpxdimen]{{bioamp-exg-pill-electrooculography-horizontal1}.jpg}
\caption{BioAmp\sphinxhyphen{}EXG\sphinxhyphen{}Pill Electrooculography\sphinxhyphen{}horizontal}\label{\detokenize{introduction/index:id7}}\end{figure}

\begin{figure}[htbp]
\centering
\capstart

\noindent\sphinxincludegraphics[width=600\sphinxpxdimen]{{bioamp-exg-pill-electrooculography-vertical1}.jpg}
\caption{BioAmp\sphinxhyphen{}EXG\sphinxhyphen{}Pill Electrooculography\sphinxhyphen{}vertical}\label{\detokenize{introduction/index:id8}}\end{figure}

\begin{figure}[htbp]
\centering
\capstart

\noindent\sphinxincludegraphics[width=600\sphinxpxdimen]{{bioamp-exg-pill-electrocardiography-Lead11}.jpg}
\caption{BioAmp\sphinxhyphen{}EXG\sphinxhyphen{}Pill Electrocardiography (ECG) Lead\sphinxhyphen{}1}\label{\detokenize{introduction/index:id9}}\end{figure}

\begin{figure}[htbp]
\centering
\capstart

\noindent\sphinxincludegraphics[width=600\sphinxpxdimen]{{bioamp-exg-pill-electroencephalography1}.jpg}
\caption{BioAmp\sphinxhyphen{}EXG\sphinxhyphen{}Pill Electroencephalography (EEG)}\label{\detokenize{introduction/index:id10}}\end{figure}


\section{Software}
\label{\detokenize{introduction/index:software}}
\sphinxAtStartPar
BioAmp EXG Pill works with any 5V microcontroller with an ADC like Arduino UNO/Nano or you can use dedicated 5v compatible ADC like ADS1115. To help with signal processing and cleaning you can use the included Arduino example sketches listed below.


\begin{savenotes}\sphinxatlongtablestart\begin{longtable}[c]{|\X{7}{92}|\X{25}{92}|\X{60}{92}|}
\hline
\sphinxstyletheadfamily 
\sphinxAtStartPar
SNO.
&\sphinxstyletheadfamily 
\sphinxAtStartPar
Program
&\sphinxstyletheadfamily 
\sphinxAtStartPar
Description
\\
\hline
\endfirsthead

\multicolumn{3}{c}%
{\makebox[0pt]{\sphinxtablecontinued{\tablename\ \thetable{} \textendash{} continued from previous page}}}\\
\hline
\sphinxstyletheadfamily 
\sphinxAtStartPar
SNO.
&\sphinxstyletheadfamily 
\sphinxAtStartPar
Program
&\sphinxstyletheadfamily 
\sphinxAtStartPar
Description
\\
\hline
\endhead

\hline
\multicolumn{3}{r}{\makebox[0pt][r]{\sphinxtablecontinued{continues on next page}}}\\
\endfoot

\endlastfoot
\begin{enumerate}
\sphinxsetlistlabels{\arabic}{enumi}{enumii}{}{.}%
\item {} 
\end{enumerate}
&
\sphinxAtStartPar
{\hyperref[\detokenize{introduction/index:}]{\emph{FixedSampling}}}
&
\sphinxAtStartPar
Sample from ADC at a fixed rate for easy processing of signal.
\\
\hline\begin{enumerate}
\sphinxsetlistlabels{\arabic}{enumi}{enumii}{}{.}%
\setcounter{enumi}{1}
\item {} 
\end{enumerate}
&
\sphinxAtStartPar
{\hyperref[\detokenize{introduction/index:}]{\emph{EMGFilter}}}
&
\sphinxAtStartPar
A 74.5 \sphinxhyphen{} 149.5 Hz band pass filter sketch for clean Electromyography.
\\
\hline\begin{enumerate}
\sphinxsetlistlabels{\arabic}{enumi}{enumii}{}{.}%
\setcounter{enumi}{2}
\item {} 
\end{enumerate}
&
\sphinxAtStartPar
{\hyperref[\detokenize{introduction/index:}]{\emph{ECGFilter}}}
&
\sphinxAtStartPar
A 0.5 \sphinxhyphen{} 44.5 Hz band\sphinxhyphen{}pass filter sketch for clean Electrocardiography.
\\
\hline\begin{enumerate}
\sphinxsetlistlabels{\arabic}{enumi}{enumii}{}{.}%
\setcounter{enumi}{3}
\item {} 
\end{enumerate}
&
\sphinxAtStartPar
{\hyperref[\detokenize{introduction/index:}]{\emph{EOGFilter}}}
&
\sphinxAtStartPar
A 0.5 \sphinxhyphen{} 19.5 Hz band\sphinxhyphen{}pass filter sketch for clean Electrooculography.
\\
\hline\begin{enumerate}
\sphinxsetlistlabels{\arabic}{enumi}{enumii}{}{.}%
\setcounter{enumi}{4}
\item {} 
\end{enumerate}
&
\sphinxAtStartPar
{\hyperref[\detokenize{introduction/index:}]{\emph{EEGFilter}}}
&
\sphinxAtStartPar
A 0.5 \sphinxhyphen{} 29.5 Hz band\sphinxhyphen{}pass filter sketch for clean Electroencephalography.
\\
\hline\begin{enumerate}
\sphinxsetlistlabels{\arabic}{enumi}{enumii}{}{.}%
\setcounter{enumi}{5}
\item {} 
\end{enumerate}
&
\sphinxAtStartPar
{\hyperref[\detokenize{introduction/index:}]{\emph{EMGEnvelop}}}
&
\sphinxAtStartPar
EMG signal envelop detection for robotics and biomedical applications.
\\
\hline\begin{enumerate}
\sphinxsetlistlabels{\arabic}{enumi}{enumii}{}{.}%
\setcounter{enumi}{6}
\item {} 
\end{enumerate}
&
\sphinxAtStartPar
{\hyperref[\detokenize{introduction/index:}]{\emph{LEDBarGraph}}}
&
\sphinxAtStartPar
LED bar graph showing EMG amplitude.
\\
\hline\begin{enumerate}
\sphinxsetlistlabels{\arabic}{enumi}{enumii}{}{.}%
\setcounter{enumi}{7}
\item {} 
\end{enumerate}
&
\sphinxAtStartPar
{\hyperref[\detokenize{introduction/index:}]{\emph{ServoControl}}}
&
\sphinxAtStartPar
Servo motor control with EMG.
\\
\hline\begin{enumerate}
\sphinxsetlistlabels{\arabic}{enumi}{enumii}{}{.}%
\setcounter{enumi}{8}
\item {} 
\end{enumerate}
&
\sphinxAtStartPar
{\hyperref[\detokenize{introduction/index:}]{\emph{HeartBeatDetection}}}
&
\sphinxAtStartPar
Standard deviation based heart beat detection algorithm.
\\
\hline\begin{enumerate}
\sphinxsetlistlabels{\arabic}{enumi}{enumii}{}{.}%
\setcounter{enumi}{9}
\item {} 
\end{enumerate}
&
\sphinxAtStartPar
{\hyperref[\detokenize{introduction/index:}]{\emph{EyeBlinkDetection}}}
&
\sphinxAtStartPar
EOG based eye blink detection.
\\
\hline\begin{enumerate}
\sphinxsetlistlabels{\arabic}{enumi}{enumii}{}{.}%
\setcounter{enumi}{10}
\item {} 
\end{enumerate}
&
\sphinxAtStartPar
{\hyperref[\detokenize{introduction/index:}]{\emph{DrowsinessDetection}}}
&
\sphinxAtStartPar
Drowsiness detection using eye blink detection.
\\
\hline\begin{enumerate}
\sphinxsetlistlabels{\arabic}{enumi}{enumii}{}{.}%
\setcounter{enumi}{11}
\item {} 
\end{enumerate}
&
\sphinxAtStartPar
{\hyperref[\detokenize{introduction/index:}]{\emph{ClawController}}}
&
\sphinxAtStartPar
Servo claw controller
\\
\hline\begin{enumerate}
\sphinxsetlistlabels{\arabic}{enumi}{enumii}{}{.}%
\setcounter{enumi}{12}
\item {} 
\end{enumerate}
&
\sphinxAtStartPar
{\hyperref[\detokenize{introduction/index:}]{\emph{EOGPhotoCaptureBLE}}}
&
\sphinxAtStartPar
EOG based photo clicking machine using ESP32 BLE
\\
\hline
\end{longtable}\sphinxatlongtableend\end{savenotes}

\sphinxAtStartPar
\sphinxstylestrong{To\sphinxhyphen{}Do:}


\begin{savenotes}\sphinxatlongtablestart\begin{longtable}[c]{|\X{7}{92}|\X{25}{92}|\X{60}{92}|}
\hline
\sphinxstyletheadfamily 
\sphinxAtStartPar
No.
&\sphinxstyletheadfamily 
\sphinxAtStartPar
Program
&\sphinxstyletheadfamily 
\sphinxAtStartPar
Description
\\
\hline
\endfirsthead

\multicolumn{3}{c}%
{\makebox[0pt]{\sphinxtablecontinued{\tablename\ \thetable{} \textendash{} continued from previous page}}}\\
\hline
\sphinxstyletheadfamily 
\sphinxAtStartPar
No.
&\sphinxstyletheadfamily 
\sphinxAtStartPar
Program
&\sphinxstyletheadfamily 
\sphinxAtStartPar
Description
\\
\hline
\endhead

\hline
\multicolumn{3}{r}{\makebox[0pt][r]{\sphinxtablecontinued{continues on next page}}}\\
\endfoot

\endlastfoot
\begin{enumerate}
\sphinxsetlistlabels{\arabic}{enumi}{enumii}{}{.}%
\item {} 
\end{enumerate}
&
\sphinxAtStartPar
BrainWaves
&
\sphinxAtStartPar
Brain waves amplitude with FFT.
\\
\hline\begin{enumerate}
\sphinxsetlistlabels{\arabic}{enumi}{enumii}{}{.}%
\setcounter{enumi}{1}
\item {} 
\end{enumerate}
&
\sphinxAtStartPar
EOGController
&
\sphinxAtStartPar
EOG based eye movement detection (left/right) to create a game controller.
\\
\hline
\end{longtable}\sphinxatlongtableend\end{savenotes}


\section{Hardware}
\label{\detokenize{introduction/index:hardware}}
\sphinxAtStartPar
BioAmp EXG Pill has been created using KiCad and all the design files can be found under {\hyperref[\detokenize{introduction/index:}]{\emph{hardware}}} folder, including production {\hyperref[\detokenize{introduction/index:}]{\emph{Gerber}}} files. Images below shows a quick overview of the hardware design.

\begin{DUlineblock}{0em}
\item[] 
\end{DUlineblock}

\begin{figure}[htbp]
\centering
\capstart

\noindent\sphinxincludegraphics[width=550\sphinxpxdimen]{{BioAmp-EXG-Pill-v1.0b-front-black}.png}
\caption{Bioamp\sphinxhyphen{}EXG\sphinxhyphen{}Pill Front}\label{\detokenize{introduction/index:id11}}\end{figure}

\begin{figure}[htbp]
\centering
\capstart

\noindent\sphinxincludegraphics[width=550\sphinxpxdimen]{{BioAmp-EXG-Pill-v1.0b-back-black}.png}
\caption{Bioamp\sphinxhyphen{}EXG\sphinxhyphen{}Pill Back}\label{\detokenize{introduction/index:id12}}\end{figure}

\begin{figure}[htbp]
\centering
\capstart

\noindent\sphinxincludegraphics[width=550\sphinxpxdimen]{{BioAmp-EXG-Pill-v1-beta-black-render}.png}
\caption{Bioamp EXG Pill v1\sphinxhyphen{}Beta\sphinxhyphen{}Black}\label{\detokenize{introduction/index:id13}}\end{figure}


\subsection{BioAmp\sphinxhyphen{}EXG\sphinxhyphen{}Pill Dimensions \& Schematic}
\label{\detokenize{introduction/index:bioamp-exg-pill-dimensions-schematic}}

\begin{savenotes}\sphinxattablestart
\centering
\begin{tabulary}{\linewidth}[t]{|T|T|}
\hline
\begin{sphinxfigure-in-table}
\centering
\capstart
\noindent\sphinxincludegraphics[width=150\sphinxpxdimen]{{BioAmp-EXG-Pill-v1.0b-dimensions}.png}
\sphinxfigcaption{Bioamp\sphinxhyphen{}exg\sphinxhyphen{}pill Dimensions}\label{\detokenize{introduction/index:id14}}\end{sphinxfigure-in-table}\relax
&\begin{sphinxfigure-in-table}
\centering
\capstart
\noindent\sphinxincludegraphics[width=450\sphinxpxdimen]{{BioAmp-EXG-Pill-v1.0b-schematic}.png}
\sphinxfigcaption{Schematic}\label{\detokenize{introduction/index:id15}}\end{sphinxfigure-in-table}\relax
\\
\hline
\end{tabulary}
\par
\sphinxattableend\end{savenotes}

\sphinxstepscope


\chapter{Applications}
\label{\detokenize{applications/index:applications}}\label{\detokenize{applications/index::doc}}\begin{enumerate}
\sphinxsetlistlabels{\arabic}{enumi}{enumii}{}{.}%
\item {} 
\sphinxAtStartPar
{\hyperref[\detokenize{applications/index:open0}]{\sphinxcrossref{\DUrole{std,std-ref}{Electrocardiography}}}}

\item {} 
\sphinxAtStartPar
{\hyperref[\detokenize{applications/index:open1}]{\sphinxcrossref{\DUrole{std,std-ref}{Electroencephalography}}}}

\item {} 
\sphinxAtStartPar
{\hyperref[\detokenize{applications/index:open2}]{\sphinxcrossref{\DUrole{std,std-ref}{Electromyography}}}}

\item {} 
\sphinxAtStartPar
{\hyperref[\detokenize{applications/index:open3}]{\sphinxcrossref{\DUrole{std,std-ref}{Electrooculography}}}}

\end{enumerate}


\section{Electrocardiography}
\label{\detokenize{applications/index:electrocardiography}}\label{\detokenize{applications/index:open0}}
\sphinxAtStartPar
We are delighted to share that BioAmp EXG Pill is among the top ten \sphinxhref{https://hackaday.com/2021/08/31/ten-winners-of-the-hackaday-prize-supportive-tech-challenge/}{Winners Of The Hackaday Prize Supportive Tech Challenge} .It was all possible because of various \sphinxhref{https://github.com/upsidedownlabs/BioAmp-EXG-Pill/tree/main/software}{Arduino example sketches} and projects that are available for the BioAmp EXG Pill which you can use to create a supportive technology. For example, you can create a mobile heart ailment detection machine, prosthetic hand controller, EOG\sphinxhyphen{}based game controller, and many more similar projects. Let’s take a look at how you can use BioAmp EXG Pill to detect heartbeats and potentially create a real\sphinxhyphen{}time heart monitor.


\subsection{\sphinxstylestrong{Detecting heartbeats with BioAmp EXG Pill \& Arduino Nano}}
\label{\detokenize{applications/index:detecting-heartbeats-with-bioamp-exg-pill-arduino-nano}}
\begin{figure}[htbp]
\centering
\capstart

\noindent\sphinxincludegraphics[width=600\sphinxpxdimen]{{bioamp-exg-pill-electrocardiography-wave}.jpg}
\caption{Bioamp\sphinxhyphen{}EXG\sphinxhyphen{}Pill Electrocardiography\sphinxhyphen{}wave}\label{\detokenize{applications/index:id11}}\end{figure}

\sphinxAtStartPar
To record ECG and detect a heartbeat in that ECG on an Arduino Nano you first need to flash the \sphinxhref{https://github.com/upsidedownlabs/BioAmp-EXG-Pill/blob/main/software/HeartBeatDetection/HeartBeatDetection.ino}{Heart Beat Detection Arduino sketch}  onto your Nano and then connect the electrodes either on your chest or hands.

\begin{sphinxadmonition}{note}{Note:}
\sphinxAtStartPar
BioAmp EXG Pill is not a certified medical device and shouldn’t be treated like one.
\end{sphinxadmonition}

\begin{DUlineblock}{0em}
\item[] 
\end{DUlineblock}


\subsection{\sphinxstylestrong{1. Chest ECG heartbeat detection}}
\label{\detokenize{applications/index:chest-ecg-heartbeat-detection}}
\sphinxAtStartPar
A very basic method for detecting heartbeats is to place the electrodes near the heart, as shown below. You just Heart have to flash the \sphinxhref{https://github.com/upsidedownlabs/BioAmp-EXG-Pill/blob/main/software/HeartBeatDetection/HeartBeatDetection.ino}{Heart beat Detection Arduino sketch} to your Nano, do the wiring, and connect the electrodes like this:
\begin{enumerate}
\sphinxsetlistlabels{\arabic}{enumi}{enumii}{}{.}%
\item {} 
\sphinxAtStartPar
IN\sphinxhyphen{} near LEFT shoulder

\item {} 
\sphinxAtStartPar
IN+ near RIGHT shoulder

\item {} 
\sphinxAtStartPar
REF above Right leg

\end{enumerate}

\begin{DUlineblock}{0em}
\item[] 
\end{DUlineblock}

\begin{figure}[htbp]
\centering
\capstart

\noindent\sphinxincludegraphics[width=600\sphinxpxdimen]{{bioamp-exg-pill-electrocardiography-Lead1}.jpg}
\caption{Wiring and Electrode placement for chest ECG}\label{\detokenize{applications/index:id12}}\end{figure}

\begin{DUlineblock}{0em}
\item[] 
\end{DUlineblock}


\subsection{\sphinxstylestrong{2. Hand EKG heartbeat detection}}
\label{\detokenize{applications/index:hand-ekg-heartbeat-detection}}
\sphinxAtStartPar
The BioAmp EXG pill is very sensitive to BioPotential signals, so you can even detect heart beats by connecting the electrodes to your hands (much like how an Apple watch records an ECG). To do this, first flash the \sphinxhref{https://github.com/upsidedownlabs/BioAmp-EXG-Pill/blob/main/software/HeartBeatDetection/HeartBeatDetection.ino}{Heart Beat Detection Arduino sketch} to your nano, and then hook up exactly the same wiring as above, and connect the electrodes like this:
\begin{enumerate}
\sphinxsetlistlabels{\arabic}{enumi}{enumii}{}{.}%
\item {} 
\sphinxAtStartPar
IN\sphinxhyphen{} on LEFT wrist

\item {} 
\sphinxAtStartPar
IN+ on RIGHT wrist

\item {} 
\sphinxAtStartPar
REF on the back of any hand.

\end{enumerate}

\begin{DUlineblock}{0em}
\item[] 
\end{DUlineblock}
\begin{quote}

\begin{figure}[htbp]
\centering
\capstart

\noindent\sphinxincludegraphics[width=600\sphinxpxdimen]{{bioamp-exg-pill-electrocardiography-hand}.jpg}
\caption{Bioamp\sphinxhyphen{}EXG\sphinxhyphen{}Pill Electrocardiography\sphinxhyphen{}hand}\label{\detokenize{applications/index:id13}}\end{figure}
\end{quote}

\begin{DUlineblock}{0em}
\item[] 
\end{DUlineblock}


\subsection{\sphinxstylestrong{References}}
\label{\detokenize{applications/index:references}}\begin{enumerate}
\sphinxsetlistlabels{\arabic}{enumi}{enumii}{}{.}%
\item {} 
\sphinxAtStartPar
Join the \sphinxhref{https://discord.gg/6aNknuBkfN}{Upside Down Labs Discord server} for chatting.

\item {} 
\sphinxAtStartPar
Subscribe to \sphinxhref{https://en.wikipedia.org/wiki/Electrooculography}{Electrooculography (EOG)} to get notified when we do live.

\end{enumerate}

\begin{DUlineblock}{0em}
\item[] 
\end{DUlineblock}


\section{Electroencephalography}
\label{\detokenize{applications/index:electroencephalography}}\label{\detokenize{applications/index:open1}}
\sphinxAtStartPar
\sphinxhref{https://en.wikipedia.org/wiki/Electroencephalography}{Electroencephalography (EEG)} is an electrophysiological monitoring method to record the brain’s electrical activity on the scalp. During the procedure, electrodes consisting of small metal discs with thin wires are pasted onto the scalp. The electrodes detect \sphinxhref{https://www.hopkinsmedicine.org/health/treatment-tests-and-therapies/electroencephalogram-eeg}{tiny electrical charges} that result from brain\sphinxhyphen{}cell activity. Those charges are then amplified to appear on a computer screen. It is typically non\sphinxhyphen{}invasive, with the electrodes placed along the scalp. The resulting signal is called an electroencephalogram, an example of which is shown below:
\begin{quote}

\begin{figure}[htbp]
\centering
\capstart

\noindent\sphinxincludegraphics[width=600\sphinxpxdimen]{{bioamp-exg-pill-electroencephalography-wave}.jpg}
\caption{Bioamp\sphinxhyphen{}EXG\sphinxhyphen{}Pill Electroencephalography\sphinxhyphen{}wave}\label{\detokenize{applications/index:id14}}\end{figure}
\end{quote}

\begin{DUlineblock}{0em}
\item[] 
\end{DUlineblock}


\subsection{\sphinxstylestrong{Creating Electroencephalograph With BioAmp EXG Pill}}
\label{\detokenize{applications/index:creating-electroencephalograph-with-bioamp-exg-pill}}
\sphinxAtStartPar
An EEG is performed using an instrument called an electroencephalograph to produce a record called an electroencephalogram. Electrodes are commonly placed on the forehead, as shown in the diagram below, when recording frontal\sphinxhyphen{}cortex EEG signals.
\begin{quote}

\begin{figure}[htbp]
\centering
\capstart

\noindent\sphinxincludegraphics[width=600\sphinxpxdimen]{{bioamp-exg-pill-electroencephalography}.jpg}
\caption{Bioamp\sphinxhyphen{}EXG\sphinxhyphen{}Pill Electroencephalography}\label{\detokenize{applications/index:id15}}\end{figure}
\end{quote}

\begin{DUlineblock}{0em}
\item[] 
\end{DUlineblock}

\sphinxAtStartPar
To record nice clean EEG signals with BioAmp EXG Pill, all you need is the Analog Serial Out sample Arduino sketch. If you want some more control we also have a \sphinxhref{https://github.com/upsidedownlabs/BioAmp-EXG-Pill/tree/main/software/FixedSampling/FixedSampling.ino}{Fixed Sampling example sketch} for recording EEG and other biopotential signals at a specific sampling rate. You can also use the \sphinxhref{https://github.com/upsidedownlabs/BioAmp-EXG-Pill/blob/main/software/EEGFilter/EEGFilter.ino}{EEG Filter example sketch} for recording EEG at a sampling rate of 256.0 Hz and a frequency of 0.5 \sphinxhyphen{} 29.5 Hz. The image below shows the fourth\sphinxhyphen{}order Butterworth IIR digital bandpass filter used in the EEG Filter example sketch.
\begin{quote}

\begin{figure}[htbp]
\centering
\capstart

\noindent\sphinxincludegraphics[width=600\sphinxpxdimen]{{bioamp-exg-pill-eegfilter}.jpg}
\caption{Bioamp\sphinxhyphen{}EXG\sphinxhyphen{}Pill EEG filter}\label{\detokenize{applications/index:id16}}\end{figure}
\end{quote}

\begin{DUlineblock}{0em}
\item[] 
\end{DUlineblock}


\subsection{\sphinxstylestrong{Frontal EEG Recording}}
\label{\detokenize{applications/index:frontal-eeg-recording}}
\sphinxAtStartPar
The video below shows a frontal electroencephalography (EEG) recording for both open\sphinxhyphen{} and closed\sphinxhyphen{}eye positions. The transition between the signals is also very clearly visible. To get an even better EEG signal, it’s recommended to use a dedicated ADC like the Texas Instruments \sphinxhref{https://www.ti.com/product/ADS1115}{ADS115 (16\sphinxhyphen{}bit)} or \sphinxhref{https://www.ti.com/product/ADS131M08}{ADS131M08 (24\sphinxhyphen{}bit)}  .On the new BioAmp EXG Pill v1.0, we recommend creating a solder bridge on the back side of the PCB to set the narrow bandpass option.

\begin{sphinxadmonition}{note}{Note:}
\sphinxAtStartPar
BioAmp EXG Pill is not a certified medical device and should not be treated like one.
\end{sphinxadmonition}


\subsection{\sphinxstylestrong{References}}
\label{\detokenize{applications/index:id3}}\begin{enumerate}
\sphinxsetlistlabels{\arabic}{enumi}{enumii}{}{.}%
\item {} 
\sphinxAtStartPar
\sphinxurl{https://en.wikipedia.org/wiki/Electroencephalography}

\item {} 
\sphinxAtStartPar
\sphinxhref{https://www.hopkinsmedicine.org/health/treatment-tests-and-therapies/electroencephalogram-eeg}{https://www.hopkinsmedicine.org/health/tests\sphinxhyphen{}and\sphinxhyphen{}therapies/electroencephalogram\sphinxhyphen{}eeg}

\item {} 
\sphinxAtStartPar
\sphinxurl{https://github.com/upsidedownlabs/BioAmp-EXG-Pill}

\end{enumerate}

\begin{DUlineblock}{0em}
\item[] 
\end{DUlineblock}


\section{Electromyography}
\label{\detokenize{applications/index:electromyography}}\label{\detokenize{applications/index:open2}}
\sphinxAtStartPar
\sphinxhref{https://en.wikipedia.org/wiki/Electromyography}{Electomyography (EMG)} is a technique for evaluating and recording the electrical activity produced by skeletal muscles. EMG is also used as a diagnostic procedure to assess the health of muscles and the nerve cells that control them (motor neurons). EMG results can reveal nerve dysfunction, muscle dysfunction, or problems with nerve\sphinxhyphen{}to\sphinxhyphen{}muscle signal transmission. The image below shows an EMG wave recorded with BioAmp EXG Pill.
\begin{quote}

\begin{figure}[htbp]
\centering
\capstart

\noindent\sphinxincludegraphics[width=600\sphinxpxdimen]{{bioamp-exg-pill-emg-wave}.jpg}
\caption{BioAmp\sphinxhyphen{}EXG\sphinxhyphen{}Pill EMG Wave}\label{\detokenize{applications/index:id17}}\end{figure}
\end{quote}

\begin{DUlineblock}{0em}
\item[] 
\end{DUlineblock}


\subsection{\sphinxstylestrong{How to Create an Electromyograph with BioAmp EXG Pill?}}
\label{\detokenize{applications/index:how-to-create-an-electromyograph-with-bioamp-exg-pill}}
\sphinxAtStartPar
EMG is performed using an instrument called an electromyograph to produce a record called an electromyogram. Common electrode placement for recording good EMG signal for hand movement is near the ulnar nerve, as shown in the diagram below.
\begin{quote}

\begin{figure}[htbp]
\centering
\capstart

\noindent\sphinxincludegraphics[width=600\sphinxpxdimen]{{bioamp-exg-pill-Electromyograph}.jpg}
\caption{Bioamp\sphinxhyphen{}EXG\sphinxhyphen{}Pill\sphinxhyphen{}Electromyograph}\label{\detokenize{applications/index:id18}}\end{figure}
\end{quote}

\begin{DUlineblock}{0em}
\item[] 
\end{DUlineblock}

\sphinxAtStartPar
To record nice clean EMG signals with BioAmp EXG Pill, all you need is the Analog Serial Out Arduino sketch. If you want more control we also have a \sphinxhref{https://github.com/upsidedownlabs/BioAmp-EXG-Pill/tree/main/software/FixedSampling/FixedSampling.ino}{Fixed Sampling Arduino sketch} that allows you to record EMG and other Biopotential signals at a specific sampling rate. You can also use the \sphinxhref{https://github.com/upsidedownlabs/BioAmp-EXG-Pill/blob/main/software/EMGFilter/EMGFilter.ino}{EMG Filter Arduino sketch} for recording EMG at a 500 Hz sampling rate and a 74.5 \sphinxhyphen{} 149.5 Hz frequency. The image below shows the 4th order bandpass Butterworth IIR digital filter used in the EMG Filter sample sketch:
\begin{quote}

\begin{figure}[htbp]
\centering
\capstart

\noindent\sphinxincludegraphics[width=600\sphinxpxdimen]{{bioamp-exg-pill-emgfilter}.jpg}
\caption{Bioamp\sphinxhyphen{}EXG\sphinxhyphen{}Pill EMG Filter}\label{\detokenize{applications/index:id19}}\end{figure}
\end{quote}

\begin{DUlineblock}{0em}
\item[] 
\end{DUlineblock}

\sphinxAtStartPar
For practical use cases, we need a smooth signal, but the raw EMG signal does not come out as a smooth curve, which is why we offer the \sphinxhref{https://github.com/upsidedownlabs/BioAmp-EXG-Pill/tree/main/software/EMGEnvelop/EMGEnvelop.ino}{envelope\sphinxhyphen{}detection Arduino sketch} for BioAmp EXG Pill. The video below shows envelope detection at work:


\subsection{\sphinxstylestrong{Example EMG Projects with BioAmp EXG Pill}}
\label{\detokenize{applications/index:example-emg-projects-with-bioamp-exg-pill}}
\sphinxAtStartPar
BioAmp EXG Pill allows you to do a lot with muscle power. We show two examples below: a servo controller and an LED bar graph. Both demonstrations use code that is derived from the \sphinxhref{https://github.com/upsidedownlabs/BioAmp-EXG-Pill/tree/main/software/EMGEnvelop/EMGEnvelop.ino}{envelope\sphinxhyphen{}detection Arduino sketch} . The image below shows how envelope detection works on the Raw EMG signal.
\begin{quote}

\begin{figure}[htbp]
\centering
\capstart

\noindent\sphinxincludegraphics[width=600\sphinxpxdimen]{{bioamp-exg-pill-emgenvelope}.jpg}
\caption{Bioamp\sphinxhyphen{}EXG\sphinxhyphen{}Pill EMG Envelope}\label{\detokenize{applications/index:id20}}\end{figure}
\end{quote}

\begin{DUlineblock}{0em}
\item[] 
\end{DUlineblock}
\begin{itemize}
\item {} 
\sphinxAtStartPar
Servo Controller

\end{itemize}

\sphinxAtStartPar
Controlling a servo motor with BioAmp EXG Pill is pretty easy, as shown in the video below. To create the same project, all you have to do is load the \sphinxhref{https://github.com/upsidedownlabs/BioAmp-EXG-Pill/tree/main/software/ServoControl/ServoControl.ino}{servo\sphinxhyphen{}control Arduino sketch} onto your Arduino\sphinxhyphen{}compatible board, and you’re good to go.
\begin{itemize}
\item {} 
\sphinxAtStartPar
LED Bar Graph

\end{itemize}

\sphinxAtStartPar
The same envelope\sphinxhyphen{}detection concept is used for the LED bar graph project. You can use as many LEDs you want with this project. To make a cool LED bar graph like the one shown in the video below, just edit the \sphinxhref{https://github.com/upsidedownlabs/BioAmp-EXG-Pill/tree/main/software/LEDBarGraph/LEDBarGraph.ino}{LED bar graph Arduino sketch} , upload it to the Arduino board, then connect the LEDs according to the specified pin numbers.


\subsection{\sphinxstylestrong{References}}
\label{\detokenize{applications/index:id5}}\begin{enumerate}
\sphinxsetlistlabels{\arabic}{enumi}{enumii}{}{.}%
\item {} 
\sphinxAtStartPar
\sphinxhref{https://www.mayoclinic.org/tests-procedures/emg/about/pac-20393913}{Mayo Clinic article on EMG}

\item {} 
\sphinxAtStartPar
\sphinxhref{https://en.wikipedia.org/wiki/Electromyography}{Wikipedia Electromyography page}

\item {} 
\sphinxAtStartPar
\sphinxhref{https://www.instructables.com/Claw-Controller-Using-BIOAMP-EXG-PILL/}{PR ROBOTICS Instructables article on Claw Controller Using BioAmp EXG PILL}

\item {} 
\sphinxAtStartPar
\sphinxhref{https://github.com/upsidedownlabs/BioAmp-EXG-Pill}{Upside Down Labs \sphinxhyphen{} BioAmp EXG Pill Github repository}

\end{enumerate}

\begin{DUlineblock}{0em}
\item[] 
\end{DUlineblock}


\section{Electrooculography}
\label{\detokenize{applications/index:electrooculography}}\label{\detokenize{applications/index:open3}}
\sphinxAtStartPar
\sphinxhref{https://en.wikipedia.org/wiki/Electrooculography}{Electrooculography (EOG)} is a technique for measuring the corneo\sphinxhyphen{}retinal standing potential that exists between the front and the back of the human eye. The resulting signal is called the electrooculogram.

\begin{figure}[htbp]
\centering
\capstart

\noindent\sphinxincludegraphics[width=600\sphinxpxdimen]{{bioamp-exg-pill-eog-demo}.jpg}
\caption{Bioamp\sphinxhyphen{}EXG\sphinxhyphen{}Pill EOG\sphinxhyphen{}Demo}\label{\detokenize{applications/index:id21}}\end{figure}

\begin{DUlineblock}{0em}
\item[] 
\end{DUlineblock}

\sphinxAtStartPar
To measure eye movement, pairs of electrodes are typically placed either above and below the eye or to the left and right of the eye. If the eye moves from center position toward one of the two electrodes, this electrode “sees” the positive side of the retina and the opposite electrode “sees” the negative side of the retina. Consequently, a potential difference occurs between the electrodes. Assuming the resting potential is constant, the recorded potential is a measure of the eye’s position.

\begin{DUlineblock}{0em}
\item[] 
\end{DUlineblock}


\subsection{\sphinxstylestrong{How to Create an Electrooculograph With BioAmp EXG Pill?}}
\label{\detokenize{applications/index:how-to-create-an-electrooculograph-with-bioamp-exg-pill}}
\sphinxAtStartPar
EOG is performed using an instrument called an electrooculograph to produce a record called an electrooculogram. Common electrode placement for recording good EOG signal for eye movement (vertical and horizontal) is shown in the diagram below.

\begin{figure}[htbp]
\centering
\capstart

\noindent\sphinxincludegraphics[width=600\sphinxpxdimen]{{bioamp-exg-pill-eog-electrode-placement}.jpg}
\caption{Bioamp\sphinxhyphen{}EXG\sphinxhyphen{}Pill EOG Electrode Placement}\label{\detokenize{applications/index:id22}}\end{figure}

\begin{DUlineblock}{0em}
\item[] 
\end{DUlineblock}

\sphinxAtStartPar
To record nice clean EOG signals with BioAmp EXG Pill, all you need is the Analog Serial Out Arduino example sketch. If you want some more control we also have a \sphinxhref{https://github.com/upsidedownlabs/BioAmp-EXG-Pill/tree/main/software/FixedSampling}{Fixed Sampling example sketch} for recording EOG and other biopotential signals at a specific sampling rate. You can also use the \sphinxhref{https://github.com/upsidedownlabs/BioAmp-EXG-Pill/blob/main/software/EOGFilter/EOGFilter.ino}{EOG Filter example sketch} for recording EOG at a sampling rate of 75.0 Hz and a frequency of 0.5 \sphinxhyphen{} 19.5 Hz. The image below shows the fourth\sphinxhyphen{}order Butterworth IIR digital bandpass filter used in the EOG Filter example sketch.

\begin{figure}[htbp]
\centering
\capstart

\noindent\sphinxincludegraphics[width=600\sphinxpxdimen]{{bioamp-exg-pill-eogfilter}.jpg}
\caption{Bioamp\sphinxhyphen{}EXG\sphinxhyphen{}Pill EOG Filter}\label{\detokenize{applications/index:id23}}\end{figure}

\begin{sphinxadmonition}{note}{Note:}
\sphinxAtStartPar
BioAmp EXG Pill is not a certified medical device and should not be treated like one.
\end{sphinxadmonition}

\begin{DUlineblock}{0em}
\item[] 
\end{DUlineblock}


\subsection{\sphinxstylestrong{Recording Horizontal Eye Movement}}
\label{\detokenize{applications/index:recording-horizontal-eye-movement}}
\sphinxAtStartPar
To record horizontal eye movement, place the electrodes near your left and right eyes, just like in the image shown below. Then you just have to flash the \sphinxhref{https://github.com/upsidedownlabs/BioAmp-EXG-Pill/blob/main/software/EOGFilter/EOGFilter.ino}{EOG filter program} onto your MCU, complete the wiring, and connect the electrodes like this:
\begin{enumerate}
\sphinxsetlistlabels{\arabic}{enumi}{enumii}{}{.}%
\item {} 
\sphinxAtStartPar
\sphinxtitleref{IN\sphinxhyphen{}} near right eye

\item {} 
\sphinxAtStartPar
\sphinxtitleref{IN+} near left eye

\item {} 
\sphinxAtStartPar
\sphinxtitleref{REF} behind ear

\end{enumerate}

\begin{DUlineblock}{0em}
\item[] 
\end{DUlineblock}
\begin{quote}

\begin{figure}[htbp]
\centering
\capstart

\noindent\sphinxincludegraphics[width=600\sphinxpxdimen]{{bioamp-exg-pill-electrooculography-horizontal}.jpg}
\caption{BioAmp EXG Pill Horizontal Eye\sphinxhyphen{}Movement}\label{\detokenize{applications/index:id24}}\end{figure}
\end{quote}


\subsection{\sphinxstylestrong{Recording Vertical Eye Movement}}
\label{\detokenize{applications/index:recording-vertical-eye-movement}}
\sphinxAtStartPar
To record vertical eye movement, place the electrodes above and below your left or right eye, just like in the image shown below. Then, just like before, flash the \sphinxhref{https://github.com/upsidedownlabs/BioAmp-EXG-Pill/blob/main/software/EOGFilter/EOGFilter.ino}{EOG filter program} onto your MCU, complete the wiring, and connect the electrodes like this:
\begin{enumerate}
\sphinxsetlistlabels{\arabic}{enumi}{enumii}{}{.}%
\item {} 
\sphinxAtStartPar
\sphinxtitleref{IN\sphinxhyphen{}} above eye

\item {} 
\sphinxAtStartPar
\sphinxtitleref{IN+} below eye

\item {} 
\sphinxAtStartPar
\sphinxtitleref{REF} behind ear

\end{enumerate}

\begin{figure}[htbp]
\centering
\capstart

\noindent\sphinxincludegraphics[width=600\sphinxpxdimen]{{bioamp-exg-pill-electrooculography-vertical}.jpg}
\caption{BioAmp EXG Pill Electrooculography\sphinxhyphen{}vertical}\label{\detokenize{applications/index:id25}}\end{figure}


\subsection{\sphinxstylestrong{Eye\sphinxhyphen{}Blink Detection}}
\label{\detokenize{applications/index:eye-blink-detection}}
\sphinxAtStartPar
Eye blinks can also be recorded with the same electrode placement. Although we used an Arduino Nano for our testing, you can flash the \sphinxhref{https://github.com/upsidedownlabs/BioAmp-EXG-Pill/blob/main/software/EyeBlinkDetection/EyeBlinkDetection.ino}{Arduino eye\sphinxhyphen{}blink detection sketch}
onto any Arduino\sphinxhyphen{}compatible board and use it to detect eye blinks. The video below shows a working demo:

\begin{DUlineblock}{0em}
\item[] 
\end{DUlineblock}


\subsection{\sphinxstylestrong{References}}
\label{\detokenize{applications/index:id9}}\begin{enumerate}
\sphinxsetlistlabels{\arabic}{enumi}{enumii}{}{.}%
\item {} 
\sphinxAtStartPar
\sphinxurl{https://en.wikipedia.org/wiki/Electrooculography}

\item {} 
\sphinxAtStartPar
\sphinxurl{https://github.com/ChinmayLonkar/MarioEMG}

\item {} 
\sphinxAtStartPar
\sphinxurl{https://github.com/upsidedownlabs/BioAmp-EXG-Pill}

\end{enumerate}

\sphinxstepscope


\chapter{Software}
\label{\detokenize{software/index:software}}\label{\detokenize{software/index::doc}}\begin{enumerate}
\sphinxsetlistlabels{\arabic}{enumi}{enumii}{}{.}%
\item {} 
\sphinxAtStartPar
{\hyperref[\detokenize{software/index:open20}]{\sphinxcrossref{\DUrole{std,std-ref}{ECG Software}}}}

\item {} 
\sphinxAtStartPar
{\hyperref[\detokenize{software/index:open21}]{\sphinxcrossref{\DUrole{std,std-ref}{EEG Software}}}}

\item {} 
\sphinxAtStartPar
{\hyperref[\detokenize{software/index:open22}]{\sphinxcrossref{\DUrole{std,std-ref}{EMG Software}}}}

\item {} 
\sphinxAtStartPar
{\hyperref[\detokenize{software/index:open23}]{\sphinxcrossref{\DUrole{std,std-ref}{EOG Software}}}}

\end{enumerate}

\begin{DUlineblock}{0em}
\item[] 
\end{DUlineblock}


\section{ECG Software}
\label{\detokenize{software/index:ecg-software}}\label{\detokenize{software/index:open20}}
\sphinxAtStartPar
This section will contain info about ECG Software

\begin{DUlineblock}{0em}
\item[] 
\end{DUlineblock}


\section{EEG Software}
\label{\detokenize{software/index:eeg-software}}\label{\detokenize{software/index:open21}}
\sphinxAtStartPar
This section will contain info about EEG Software

\begin{DUlineblock}{0em}
\item[] 
\end{DUlineblock}


\section{EMG Software}
\label{\detokenize{software/index:emg-software}}\label{\detokenize{software/index:open22}}
\sphinxAtStartPar
This section will contain info about EMG Software

\begin{DUlineblock}{0em}
\item[] 
\end{DUlineblock}


\section{EOG Software}
\label{\detokenize{software/index:eog-software}}\label{\detokenize{software/index:open23}}
\sphinxAtStartPar
This section will contain info about EOG Software

\begin{DUlineblock}{0em}
\item[] 
\end{DUlineblock}

\sphinxstepscope


\chapter{Hardware}
\label{\detokenize{hardware/index:hardware}}\label{\detokenize{hardware/index::doc}}\begin{enumerate}
\sphinxsetlistlabels{\arabic}{enumi}{enumii}{}{.}%
\item {} 
\sphinxAtStartPar
{\hyperref[\detokenize{hardware/index:open10}]{\sphinxcrossref{\DUrole{std,std-ref}{Arduino}}}}

\item {} 
\sphinxAtStartPar
{\hyperref[\detokenize{hardware/index:open11}]{\sphinxcrossref{\DUrole{std,std-ref}{ESP32}}}}

\item {} 
\sphinxAtStartPar
{\hyperref[\detokenize{hardware/index:open12}]{\sphinxcrossref{\DUrole{std,std-ref}{RPi Pico}}}}

\item {} 
\sphinxAtStartPar
{\hyperref[\detokenize{hardware/index:open13}]{\sphinxcrossref{\DUrole{std,std-ref}{STM32}}}}

\item {} 
\sphinxAtStartPar
{\hyperref[\detokenize{hardware/index:open14}]{\sphinxcrossref{\DUrole{std,std-ref}{BBBW}}}}

\end{enumerate}

\begin{DUlineblock}{0em}
\item[] 
\end{DUlineblock}


\section{Arduino}
\label{\detokenize{hardware/index:arduino}}\label{\detokenize{hardware/index:open10}}
\sphinxAtStartPar
This section will contain info about using Bio\sphinxhyphen{}Amp\sphinxhyphen{}EXG\sphinxhyphen{}Pill with arduino.

\begin{DUlineblock}{0em}
\item[] 
\end{DUlineblock}


\section{ESP32}
\label{\detokenize{hardware/index:esp32}}\label{\detokenize{hardware/index:open11}}
\sphinxAtStartPar
This section will contain info about using Bio\sphinxhyphen{}Amp\sphinxhyphen{}EXG\sphinxhyphen{}Pill with ESP32.

\begin{DUlineblock}{0em}
\item[] 
\end{DUlineblock}


\section{RPi Pico}
\label{\detokenize{hardware/index:rpi-pico}}\label{\detokenize{hardware/index:open12}}
\sphinxAtStartPar
This section will contain info about using Bio\sphinxhyphen{}Amp\sphinxhyphen{}EXG\sphinxhyphen{}Pill with RPi Pico.

\begin{DUlineblock}{0em}
\item[] 
\end{DUlineblock}


\section{STM32}
\label{\detokenize{hardware/index:stm32}}\label{\detokenize{hardware/index:open13}}
\sphinxAtStartPar
This section will contain info about using Bio\sphinxhyphen{}Amp\sphinxhyphen{}EXG\sphinxhyphen{}Pill with STM32 Blue Pill.

\begin{DUlineblock}{0em}
\item[] 
\end{DUlineblock}


\section{BBBW}
\label{\detokenize{hardware/index:bbbw}}\label{\detokenize{hardware/index:open14}}
\sphinxAtStartPar
This section will contain info about using Bio\sphinxhyphen{}Amp\sphinxhyphen{}EXG\sphinxhyphen{}Pill with bbbw board.

\begin{DUlineblock}{0em}
\item[] 
\end{DUlineblock}



\renewcommand{\indexname}{Index}
\printindex
\end{document}